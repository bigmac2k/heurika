 \section{Usage}
 \subsection{Starting the tool}
    To use the Heurika tool one has to follow a few steps.
    First of all, one needs to specify a configuration file.
    The file has to be written based on the following grammar:\\	
	\begin{tabular}{p{3cm}l}
	    File &= Header Sep Nodes Sep Edges Sep Screens\\\\
	    
	    Header &= PdfFile``\textbackslash n''NumScreens``\textbackslash n''Start``\textbackslash n''SpeakerTracks\\
	    PdfFile &= ``PdfFile '' PDF\\
	    NumScreens &= ``NumScreens '' Digit\\
	    Start &= ``Start '' NodeId\\
	    SpeakerTracks &= ``SpeakerTracks '' Digit ``\textbackslash n''\\\\
	    
	    Nodes &= (``Node '' NodeId Node$^+$ ``\textbackslash n'')$^+$\\
	    
	    Edges &= Edge ``\textbackslash n'' $|$ Edge ``\textbackslash n''Edge ``\textbackslash n''\\
	    Edge &= Node (Dir ``$>$'' NodeId $|$ ``$<$'' Dir NodeId)$^+$ \\
	    &$\;\;$ $|$ NodeId ``$<$lr$>$'' NodeId$^+$ $|$ NodeId ``$<$lr$>$'' NodeId ``..'' NodeId\\
	    &$\;\;$ $|$ NodeId ``$<$ud$>$'' NodeId$^+$ $|$ NodeId ``$<$ud$>$'' NodeId ``..'' NodeId\\
	    
        Screen &= (``Screen'' ((String ``('' Digit``)''$)\|$(String$)^+)^+$\\\\
	    
        
        Dir &= $``r'' | ``l'' | ``u'' | ``d''$\\
        PDF &= String``.pdf''\\
        Node &= Digit\\
        NodeId &= Digit\\
        Sep &= ``;\textbackslash n''\\
        String &= $ [a-zA-Z]^+ $ \\
        Digit &= $[1-9][0-9]^*$ \\ 
	    
    \end{tabular}\\
    
    In the header of the configuration file, one has to specify the PDF file.
    The tool only accepts one PDF file.
    Therefore, in case one does not only want to show the content slides but also notes, the notes need to be appended to the content slides in the PDF file.
    Besides the PDF file, the number of screens has to be defined.
    A probable configuration of the screens could be, one screen for the beamer, another with the notes on the local/native screen and another one for the web server.
    
    The start attribute determines which node will be shown on all screens during startup.
    In case, one sets the start attribute to 1, the node with the id 1 will be shown on all screens during startup.
    The speaker track defines, which tracks will be just available with a password on the web server.
    Hence, the header of the configure file could be defined in the following manner:
    \begin{verbatim}
    PdfFile path/fp.pdf
    NumScreens 3
    Start 1
    SpeakerTracks 3
    ;
    \end{verbatim}
    
    Nodes define which slide will be shown on what screen/track.
    The first attribute of a node is an id.
    After the id, the PDF slide has to be specified, which will be shown a track for that node id.
    Furthermore, there need to be exactly as many tracks as screens.
    Thus, there should be as many columns as screens are defined.
    Counting the id as well, there are number of screens + 1 columns after the Node keyword.
    Let us assume, we have a PDF file with 5 content slides and 5 note slides.
    Further, we have two screens and a web server.
    Then we would have node ids from 1 to 5.
    The first track would show the pages 1 to 5, namely the slides and the second track would show the pages 6 to 10, namely the notes.
    Lastly, the third track for the web server would look identical to the second track.
    Therefore, the node part of the configure file would look like this:
    \begin{verbatim}
    Node 1 1 6 6
    Node 2 2 7 7
    Node 3 3 8 8
    Node 4 4 9 9
    Node 5 5 10 10
    ;
    \end{verbatim}
    
    Now, the next important part is to define how one can switch between the node ids.
    Mostly, one will probably have a main path through the slides.
    The main path can be most easily defined by stating with first node id, then $<$ud$>$ and the second node id.
    After the second node id a ``..'' and the last node id of the path.
    Thereby, one defines a start point, an interval and an end point.
    The $<$ud$>$ means that one can go from the left node to the right by pressing down and the other way around by pressing up.
    Besides the main path, one can have side paths with additional information.
    One could for example go from the 3rd slide to the 5th by pressing the left arrow.
    The side path could be defined by stating 3 l$>$ 5.
    Further, one would be able to go forward from the 5 slide to the 4, 5 d$>$4.
    The whole transition rules would be:
    \begin{verbatim}
    1 <ud> 2 .. 4
    3 l> 5
    5 d> 4
    ;
    \end{verbatim}
    
    Lastly, one needs to specify the screens.
    For the depicted configuration it is most likely to have two native screens, one for the slides on the beamer and another with the nodes on the own display.
    The first track will also be shown on the web server for the audience.
    The last track would be on the web server.
    Here we specified, that the port 5001 will be used for the web server with the slides on it.
    Further, the web server with the notes will be on port 5000.
    \begin{verbatim}
    Screen newnative webserver(5001)
    Screen newnative
    Screen webserver(5000)
    \end{verbatim}
    
    The whole file would then be:
    \begin{verbatim}
    PdfFile path/fp.pdf
    NumScreens 3
    Start 1
    SpeakerTracks 1
    ;
    Node 1 1 6 1
    Node 2 2 7 2
    Node 3 3 8 3
    Node 4 4 9 4
    Node 5 5 10 5
    ;
    1 <ud> 2 .. 4
    3 l> 5
    5 d> 4
    ;
    Screen newnative webserver(5001)
    Screen newnative
    Screen webserver(5000)
    \end{verbatim}
    
    Having defined the configuration file one can use the tool.
    The configuration file needs to be passed to the tool as the first argument.
    Besides the file one has to specify, in what radius the tool should pre-render the slides.
    By passing the -s command with a number one specifies the radius.
    Further, one can specify the password for the speaker track on the web server with -p.
    To specify the size of the rendered image, which will be show one has to specify them with -x and -y.
    One should enter the size of the screen with the highest resolution here.
    The default resolution is 1024x768.
    Besides the size, one can also change the quality of the images with -q.
    The quality is a number between 1 and 100, to specify how good the quality of the jpeg should be compared to the PDF.
    The default quality is 90.
    In case one only wants to have a normal PDF viewer, one simply has to pass -presenter with the file to be presented.
    Further, there is a --help to see all possible commands.
    \subsection{Native viewer}
    Now, the tool will open and based on the native screens defined windows will open.
    To make the window fullscreen just press f and the window will fullscreen on the screen it is on.
    In case, one has a track with slides and another one with notes, the user will have to drag the window with the slides to the beamer and leave the notes on the native display.
    Afterwards, both can be set to fullscreen.
    
    Now, the slides can be controlled by the arrow keys, defined by the edges in the configure file.
    To go back one slide, one can press press ``Escape''.
    However, one can not only go back to the last slide, but also to the last time one left a main path.
    Leaving a main path in this case means going either right or left.
    One can go back to the last main path by pressing the ``right Alt'' key.
    Further, the slides will change in both windows, when pressing the key once.
    The tool can be closed by pressing q in any window. 