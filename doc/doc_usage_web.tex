\subsection{Web interface}
Having a track for the web server, one can reach the web server via the IP of the machine HeuRIKA is running on. 
The port as it is defined in the configuration file has to be added. 
In the most simple case it could look like this:
\begin{verbatim}
http://192.168.0.102:5000
\end{verbatim}

In case, there is no public track defined, the visitor will be redirected to a password prompt.
Type in "speaker" as the username and the password from the command line. 
The slides will be displayed in the upper part and controls in the lower part.
If there is a public track defined, calling the URL will show the current slide as it is defined in the configuration file.
Below you find an explanation of the controls.
\setlength{\tabcolsep}{10pt}
\begin{table}[htbp]
	\centering
		\begin{tabular}{lcl}
			Stackback & \includegraphics[scale=0.05]{images/arrow_stackback_active.png} & Do a jump to the begin of the current branch.\\
			\\
			Left & \includegraphics[scale=0.05]{images/arrow_left_active.png} & Enter the branch to the left.\\
	\\
			Up & \includegraphics[scale=0.05]{images/arrow_up_active.png} & Go one slide up the tree.\\
	\\
			Down & \includegraphics[scale=0.05]{images/arrow_down_active.png} & Go one slide down the tree.\\
	\\
			Back & \includegraphics[scale=0.05]{images/arrow_back_active.png} & Go one slide back.\\
	\\
			Right & \includegraphics[scale=0.05]{images/arrow_right_active.png} & Enter the branch to the right.\\
		\end{tabular}
\end{table}

When there is another track defined, one can switch to this by clicking the link in the footer of the page.

On the Apple iPad and iPhone there is a special opportunity to get a fullscreen display of the page: 
By adding a link to the homescreen via the menu and clicking this link, the page is displayed in fullscreen mode.